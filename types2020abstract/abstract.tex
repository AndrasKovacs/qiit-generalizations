
\documentclass{easychair}

\usepackage{doc}
\usepackage{scalerel}
\usepackage{amsfonts}

\newcommand{\easychair}{\textsf{easychair}}
\newcommand{\miktex}{MiK{\TeX}}
\newcommand{\texniccenter}{{\TeX}nicCenter}
\newcommand{\makefile}{\texttt{Makefile}}
\newcommand{\latexeditor}{LEd}
\newcommand{\emptycon}{\scaleobj{.75}\bullet}
\newcommand{\ext}{\triangleright}
\newcommand{\arri}{\Rightarrow}

\newcommand{\ToS}{\mathsf{ToS}}
\newcommand{\U}{\mathsf{U}}

\title{Generalizations of Quotient Inductive-Inductive Types %
  \thanks{Both authors author were supported by the European Union, co-financed
    by the European Social Fund (EFOP-3.6.3-VEKOP-16-2017-00002). The second
    author was supported by by the National Research, Development and Innovation
    Fund of Hungary, financed under the Thematic Excellence Programme funding
    scheme, Project no.\ ED\_18-1-2019-0030 (Application-specific highly
    reliable IT solutions). }}

\author{
Andr\'as Kov\'acs
\and
Ambrus Kaposi
}

% Institutes for affiliations are also joined by \and,
\institute{
  E\"otv\"os Lor\'and University,
  Budapest, Hungary \\
  \email{kovacsandras|akaposi@inf.elte.hu}
}

\authorrunning{Kov\'acs, Kaposi}
\titlerunning{Generalizations of Quotient Inductive-Inductive Types}
\pagenumbering{gobble}
\begin{document}

\maketitle

Quotient inductive-inductive types (QIITs) are the most general class of
inductive types studied thus far in a set-truncated setting, i.e.\ in the
presence of uniqueness of identity proofs (UIP). In the current work, we develop
QIITs further, focusing on applications in practical metatheory of type
theories. We extend previous work on QIITs \cite{kaposi2019constructing} with
the following:
\begin{enumerate}
  \item
  \textbf{Large constructors, large elimination} and algebras at different
  universe levels. This fills in an important formal gap; large models are
  routinely used in the metatheory of type theories, but they have not been
  presented explicitly in previous QIIT literature.
  \item
  \textbf{Infinitary constructors}.
  This covers real, surreal \cite{HoTTbook} and ordinal numbers. Additionally,
  the theory which describes QII signatures is itself a large and infinitary
  QIIT, which allows the theory of signatures to describe its own signature
  (modulo universe levels), and provide its own model theory. This was not
  possible previously in \cite{kaposi2019constructing}, where only finitary
  QIITs were described.
  \item
  \textbf{Recursive equations}, i.e. equations appearing as assumptions of
  constructors. These have occurred previously in syntaxes of cubical type
  theories, as boundary conditions \cite{cohen2016cubical,
    angiuli2016computational, angiuli2018cartesian}.
  \item
  \textbf{Sort equations}. Sort equations were included in Cartmell's
  generalized algebraic theories (GATs) \cite{gat}, which overlap significantly
  with finitary QIITs. Sort equations appear to be useful for algebraic
  presentations of Russell-style and cumulative universes
  \cite{sterling2019algebraic}.
\end{enumerate}

\subsubsection*{Self-describing signatures} In the current work, we
would also like to streamline and make more rigorous the specification of
signatures. Previous descriptions of GATs \cite{gat, sterling2019algebraic} used
raw syntax with well-formedness relations to describe signatures, which is
rather unwieldy to formally handle. Also, the precursor of the current work
\cite{kaposi2019constructing} used an ad-hoc QIIT to describe signatures, which
did not have a model theory worked out, and its existence was simply assumed.

In contrast, equipped with large elimination and self-description, we are able
to specify signatures and develop a model theory for signatures, without ever
using raw syntax or assuming the existence of a particular QIIT. We do the
following in order.
\begin{enumerate}
  \item
  We specify a notion of model for the theory of signatures (ToS); this is a
  category with family (CwF) extended with several type formers, allowing to
  represent a signature as a typing context, with types specifying various
  constructors.
  \item We say that a signature is a context in an \emph{arbitrary} model, i.e.
    a function with type $(M : \mathsf{ToS})\to\mathsf{Con}_M$. This can be
    viewed as a fragment of a Church-encoding; here we do not care about
    encoding the whole syntax of ToS, nor the initiality of the syntax, we only
    need a representation of signatures and the ability to interpret a signature
    in a $\mathsf{ToS}$-model. For example, the signature for natural number
    algebras is a function
    \[
    \lambda(M : \mathsf{ToS}).\,(\emptycon_M\,\ext_M\,(N : \mathsf{U}_M)\,\ext_M\,(zero : \mathsf{El}_M\,N)\,\ext_M\,(suc : N\arri_M\,\mathsf{El}_M\,N))
    \]
    which maps every model $M$ to a typing context in $M$, consisting of the
    declaration of a sort and two constructors.

  \item We give a semantics for signatures, as a particular $M : \mathsf{ToS}$
    model which interprets each CwF context $\Gamma$ as a structured category of
    $\Gamma$-algebras. E.g.\ for the signature of natural numbers, we get a
    structured category of $\mathbb{N}$-algebras, with
    $\mathbb{N}$-homomorphisms as morphisms.
  \item We give a (large, infinitary) signature for $\mathsf{ToS}$ itself, such
    that interpreting the signature in the semantic model yields a structured
    category of $\mathsf{ToS}$-algebras. From this, we acquire notions of
    \emph{recursion} and \emph{induction}, hence we gain the ability to define
    further constructions by induction on an assumed initial model of the theory
    of signatures.
\end{enumerate}
In the above construction, everything is appropriately indexed with universe
levels (we omit the details), and there is a ``bump'' of levels at every
instance of self-description.

\subsubsection*{Extending semantics to infinitary constructors and sort equations.}
Previously in \cite{kaposi2019constructing}, contexts in $\ToS$ were interpreted
as CwFs of algebras with extra structure, substitutions as strict morphisms of
such CwFs, and types as displayed CwFs. Infinitary constructors force a major
change: substitutions must be interpreted as weak CwF morphisms, and types as
CwF isofibrations, which are displayed CwFs with an additional lifting structure
for isomorphisms. In short, this means that the semantics of infinitary
constructors can be only given mutually with a form of invariance under algebra
isomorphisms. Recursive equations similarly require this kind of semantics.

However, strict sort equations are not invariant under isomorphisms. For
example, if we have an isomorphism in $\mathbf{Set}\times\mathbf{Set}$ between
$(A, B)$ and $(A', B')$, and we also know that $A = B$ strictly, then it is not
necessarily the case that $A' = B'$. This means that a strict semantics for sort
equations is incompatible with the isofibration semantics for infinitary
constructors. Our current solution is to simply keep the troublesome features
apart. Hence we have
\begin{enumerate}
\item A theory of signatures supporting recursive equations and infinitary
  constructors, but no sort equations. This $\ToS$ can describe itself, and by a
  term model construction we can reduce all described QIITs to an assumed syntax
  of the same $\ToS$. This term model construction is also weakened (i.e.\ it is
  up to algebra isomorphisms), hence it is significantly more complicated than
  in \cite{kaposi2019constructing}.
\item A theory of signatures supporting sort equations, but no recursive
  equations and infinitary constructors. This $\ToS$ is infinitary and has no
  sort equations, so we can give it a model theory as an infinitary QIIT. This
  $\ToS$ supports a stricter semantics which is not invariant under
  isomorphisms, and we also have a term model construction. Here, the semantics
  and the term models are straightforward extensions of
  \cite{kaposi2019constructing}.
\end{enumerate}

\bibliographystyle{plain}
\bibliography{references}


\end{document}
